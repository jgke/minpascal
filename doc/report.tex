\documentclass{article}
\usepackage{amsmath}
\usepackage{amsfonts}
\usepackage{graphicx}
\usepackage{placeins}
\usepackage[top=1in,bottom=1in]{geometry}
\usepackage{minted}
\usepackage{hyperref}

\def\bnflexer{grammar_notation.py:AbnfLexer -x}
\def\pascallexer{pascal.py:DelphiLexer -x}

\newcommand{\bnf}[1]{%
    \inputminted{\bnflexer}{#1}}

\setminted{autogobble}

\author{Jaakko Hannikainen - Compilers}
\title{Miniplc, an interpreter for the MiniPL programming language}

\begin{document}
\maketitle

\noindent
\begin{minipage}{0.35\textwidth}
\begin{minted}{\pascallexer}
    program Hello;
    begin
        writeln ("Hello, World!");
    end.
\end{minted}
\end{minipage}
\begin{minipage}{0.54\textwidth}
\begin{minted}{text}
    $ java -jar build/libs/minpascal-fatjar.jar helloWorld.mpp
    $ cc helloWorld.c -o hello
    $ ./hello
    Hello, World!
    $
\end{minted}
\end{minipage}

\section{Introduction}
MinPascal is a complier for the Mini-Pascal programming language. It
compiles Mini-Pascal to a limited subset of C. \\[1.5em]

\noindent
Basic usage:
\begin{minted}{shell}
    ./gradlew jar
    java -jar build/libs/minpascal.jar helloWorld.mpp # outputs helloWorld.c
    cc helloWorld.c -o hello # Compile using your favourite C compiler and...
    ./hello # execute it!
\end{minted}

\noindent
Running tests:
\begin{minted}{shell}
    ./gradlew test # normal tests
    ./gradlew pitest # mutation tests
    ./gradlew test pitest jacocoTestReport # generate code coverage
\end{minted}

\newpage
\section{Code layout}
The compiler includes a generic AST parser in the package
\texttt{fi.jgke.minpascal.astparser}, which uses the regular expression
implementation in \texttt{fi.jgke.minpascal.util}. The auto-generated AST tree
is passed to the compiler in \texttt{fi.jgke.minpascal.compiler}, which outputs
a low-level variant of C. The compiler uses the following libraries:
\begin{itemize}
    \item Various implementations of \texttt{java.util.Collection} (such as lists, sets and maps)
    \item \texttt{java.util.stream.Stream}
    \item \texttt{java.util.function} for various Java 8 lambdas
    \item \texttt{lombok} for automatic generation of getters and setters
    \item \texttt{java.io} and \texttt{java.nio} for file input/output
    \item \texttt{com.google.common.collect.Streams}' \texttt{.zip()} function
\end{itemize}

The tests are a bit more lenient in terms of using the standard library, and
they use the standard Java implementation for regular expressions in quite many
places.

\subsection{Regular expressions}
The class \texttt{fi.jgke.minpascal.util.Regex} contains a small Regex-like language
implementation with the following properties:
\begin{itemize}
    \item Only checking for matches in the beginning of a string is supported
    \item Only does a limited amount of backtracking is done before not
        matching a string \\
        (regex such as '(x+x+)+y' doesn't match 'xxxxxxy')
    \item The following regular expression features are available:
        Concatenation (ab), alternation (a|b), kleene star (a*), lazy
        kleene star (a*?, required due to limited backtracking), kleene plus
        (a+, equivalent to aa*), groups ([a-z]), negative groups ([\^{}a-z])
        and end-of-file (\$).
    \item The regex matcher only returns the length of the match.
\end{itemize}

The regexes are first compiled into a tree of patterns, which are then used to
match strings.

\subsection{AST parser}
The AST parser has a very similar layout to the regex implementation, except
that it's broken into multiple files. The BNF file is first broken into lines,
then tokenized and parsed into matchers. These matchers can then be used to
transform text into an AST. The different types of AST nodes are located in
\texttt{fi.jgke.minpascal.astparser.nodes}, and the matchers are located in
\texttt{fi.jgke.minpascal.astparser.parsers}. Whitespace is handled by the
special rule \texttt{whitespace}, which matches on spaces, newlines and
comments.

\FloatBarrier

\noindent
\begin{figure}[ht!]
    \begin{minted}{java}
        private static List<String> tokenize(List<String> strs, String delimiter) {
            return strs.stream()
                .map(str -> Arrays.stream(str.split(Pattern.quote(delimiter), -1)))
                .flatMap(ss -> ss
                    .flatMap(s1 -> Stream.of(delimiter, s1))
                    .skip(1))
                .filter(s -> !s.trim().isEmpty())
                .collect(Collectors.toList());
        }
    \end{minted}
    \caption{Tokenizer for the BNF rules is very simple.}
\end{figure}

\FloatBarrier

\subsection{AST}
The AST is represented as a collection of AST nodes. Each node can either be
empty (eg. an optional node which didn't match), a leaf node (eg. a terminal)
or a list (from either concatenation or alternation). Alternation can be handled
in the compiler with the \texttt{toMap} helper, which forces the compiler to
check all alternatives. A backdoor (\texttt{getFirstChild}) is provided, which
can be used to check whether a node exists as an immediate child.

\begin{figure}[ht!]
    \begin{minted}{java}
        initRules("X ::= a | b",
                  "a ::= 'a'",
                  "b ::= 'b'");
        new RuleParser("X").parse("a").getLeft().toMap()
                .map("a", _ -> null)
                .map("c", _ -> null)
                .unwrap();
    \end{minted}
    \caption{A compiler exception is thrown on line 6 -- the rule for X doesn't
    include any 'c' so the compiler cannot map over it.}
\end{figure}

\begin{figure}[ht!]
    \begin{minted}{java}
        initRules("X ::= a | b",
                  "a ::= 'a'",
                  "b ::= 'b'");
        new RuleParser("X").parse("a").getLeft().toMap()
                .map("a", _ -> null)
                .unwrap();
    \end{minted}
    \caption{A compiler exception is thrown on line 6 -- 'b' isn't handled.}
\end{figure}

\newpage
The AST is generated automatically based on the BNF grammar. Due to technical
reasons, alternation matches result in duplicated AST nodes (eg.
SimpleStatement in the sample below). The following is an example of the AST
for the hello world program:

\inputminted{text}{ast.txt}


\newpage
\section{BNF for the language}
\label{sec:bnf}
\bnf{../src/main/resources/MinPascal.bnf}

\newpage
\section{Language implementation-level decisions}
Various implementation-level decisions, in no particular order:

\begin{itemize}
    \item True and false are handled as variables. This simplifies the grammar,
        while allowing overriding the identifiers.
    \item Side-effects' evaluation order is left-to-right.
    \item Integers are represented with C's \texttt{int}, and reals with
        \texttt{double}.
    \item \texttt{main()} in C always returns 0 (unless an assertion fails).
    \item Functions are required to hane 'return' as their last statement. As
        the compiler doesn't perform any logic related to handling if/while
        statements, it cannot check whether a function returns for statements
        such as if(...) return foo else return bar.
    \item Array size is stored in index -1. This makes it easy to pass around
        in C.
    \item Array index accesses are checked at runtime, and they throw assertion
        failures.
    \item Boolean type seemed to be missing from the specification (by
        accident?), and is available under the name \texttt{boolean}.
    \item The compiled C code leaks a lot of memory, due to the deadline. There
        is some support for releasing memory, but it's only used within
        \texttt{writeln} statements.
    \item Nested functions are only supported if the C compiler supports them.
        Again, deadlines.
    \item Array types can be used directly without the index access ([]), this
        is equivalent to accessing the first item ([0]).
\end{itemize}

\FloatBarrier

\begin{figure}[ht!]
    \begin{minted}{\pascallexer}
        program Hello;
        begin
          writeln("Foo" + "Bar");
        end.
    \end{minted}
    \caption{This snippet doesn't leak memory...}
\end{figure}

\FloatBarrier

\begin{figure}[ht!]
    \begin{minted}{\pascallexer}
        program Hello;
        begin
          var x: string;
          var y: string;
          x := "FooBar";
          y := "FooBar";
          writeln(x + y);
        end.
    \end{minted}
    \caption{...but this one does.}
\end{figure}

\FloatBarrier

\newpage
\section{Semantic analysis}
The execution starts by compiling the BNF rules, which are ad-hoc checked.
The actual program checks start when the input file is passed to the compiler.
Each parser has its own checks. For example, while parsing
\texttt{VarDeclaration}, the AndParser first checks whether the rule \texttt{op}
matches. The RuleMatch for \texttt{op} checks whether the input starts with
'('. The AndMatch proceeds to the next rule, which is \texttt{identifier}. This
continues, until either the whole match is complete or one of the parsers
throws an error.

After the parsing is complete, the AST tree is passed to the compiler. Most of
the errors here are either type errors or identifier-related errors. Type errors
happen when the Mini-Pascal program tries to do an unsupported operation, such
as calling an identifier that isn't a function or trying to add booleans
together. The other time type errors can be thrown is return type checking.
Identifiers are handled in the typical way; every time a new block is opened, a
map is pushed to a global stack. The stack is searched every time the program
uses any identifiers.

\section{Code generation problems and solutions}
The AST autoparser, unsurprisingly, took a lot of effort to get running. I'm
still pretty happy about how it turned out.

Pointer behaviour in Mini-Pascal is interesting, as it's implicit
(\texttt{x~+~var~y~=~x~+~*y}), and is a source of many bugs. I'm pretty sure
there are several bugs in the corner cases behind arrays, var types and the
like...

Code generation in general turned out to be a bit ugly -- it could certainly
be prettified.

The compiler doesn't try to recover from errors. It could certainly be a bit
better, at least for the parsing.

\section{Error handling strategies and solutions}

Errors are handled by throwing appropriate exceptions when errors happen. At
all times while compiling a position context is available, which can be used
to approximate the location in the file where the error happened. 

\begin{figure}[ht!]
    \begin{minted}{text}
        $ java -jar build/libs/minpascal-fatjar.jar helloWorld.mpp
        Error near line 2, column 1: Parse error: could not match 'begin'
        for string 'baegin   var x: string;  var y: strin...'
        $
    \end{minted}
    \caption{A parse error; the parser expects 'begin' but only finds 'baegin'.}
    \label{fig:parseerror}
\end{figure}

\begin{figure}[ht!]
    \begin{minted}{text}
        $ java -jar build/libs/minpascal-fatjar.jar helloWorld.mpp
        Error near line 4, column 2: Identifier x already exists,
        original defined at line 3, column 2
        $
    \end{minted}
    \caption{A duplicate identifier; the identifier 'x' is defined at line 3,
    but is redefined in the same scope on line 4.}
    \label{fig:dupe}
\end{figure}

Various checks are done while compiling; these include both parsing errors (such
as the parse error in figure~\ref{fig:parseerror}) and semantic errors (such as
type errors or the duplicate identifier in figure~\ref{fig:dupe}).

The following types of errors are handled:
\begin{itemize}
    \item Parse errors (string literal or regex didn't match, none of
        alternatives in an OrParser matched)
    \item Duplicate identifiers in the same scope (nested scope is fine)
    \item Unknown identifiers
    \item Invalid operators for types (such as 'true + true')
    \item Generic type errors (returning an invalid type, assigning an invalid
        type, calling a non-function...)
    \item Compiler bugs, such as invalid handling of AST nodes
\end{itemize}

The compiler doesn't try to recover from errors.

\end{document}
